\documentclass[twocolumn]{article}

\usepackage[british]{babel}
\usepackage{fancyhdr}
\RequirePackage[a4paper, left=2cm, right=2cm, bottom=3cm, top=2.3cm, headsep=100pt]{geometry}
\usepackage{header}
\RequirePackage[small]{titlesec} % Taille des sections réduite

\author{
  Arthur \textsc{Adjedj}\\
  Vincent \textsc{Lafeychine} \and
  Augustin \textsc{Albert} \\
  Lucas \textsc{Tabary-Maujean}
}

\title{
  \includegraphics[height=2.5cm]{media/logo}

  \textbf{Proost: specifications}\\
  \large A small proof assistant written in Rust
  \\[1\baselineskip]\normalsize ENS Paris-Saclay
}


%%% headings
\pagestyle{fancy}
\renewcommand{\headrulewidth}{0pt}
\renewcommand{\footrulewidth}{0pt}

\fancyhead{}
\fancyfoot[L]{\small{\reflectbox{\copyright} \the\year{}}}


%%%
\setlength{\columnsep}{20pt} % 10pt is the default

\newcommand{\members}[1]{\texorpdfstring{\hfill\scriptsize #1}{}}

\newcommand{\etun}{{\color{Green} ($\star$)} }
\newcommand{\etde}{{\color{Orange} ($\star\star$)} }


%%%
\begin{document}
\thispagestyle{fancy}
\maketitle

\emph{
	This project is under development, and its specifications themselves are
	subject to changes, should time be an issue or a general consensus be reached
	to change the purposes of the tool. An example of that is the syntax of the
	language, which is still largely unstable. }

\section{General purpose and functions}
This project aims at providing a small tool for typechecking expressions written
in the language of the Calculus of Construction (CoC). This tool shall be both
terminal and editor based through the \texttt{proost} program that provides both
compiler and toplevel-like capacities and options and a LSP called
\texttt{tilleul}. The file extension used by both programs is \texttt{.mdln},
which is short for \emph{madeleine}, the name of the language manipulated by
users in these files.


\section{Project structure}
Each category of the project is assigned some or all members of the group,
meaning the designated members will \emph{mainly} make progress in the
associated categories and review the corresponding advancements. Any member may
regardless contribute to any part of the development of the tool.

Some specific categories and items will be added a star \etun or two \etde to
indicate whether they are respectively current objectives or extra long-term
requirements that may or may not be implemented.


\subsection{Language design \members{all members}} The \texttt{proost} tool is a
simple proof assistant and does not provide any tactics. As new features arrive
from extensions of the kernel type theory, the \emph{madeleine} language must
provide convenient shorthands and notations. The syntax of commands is the
following:
\begin{itemize}
	\item ¤import relative_path_to_file¤ typechecks
	      and loads the file in the current environment;
	\item \etun ¤search t¤ searches ¤t¤ in all known terms, where ¤t¤ may contain
	      ¤_¤ joker identifiers (see section \ref{sec:unification});
	\item ¤def a := t¤
	      defines an alias ¤a¤ that can be used in any following command;
	\item ¤def a: ty := t¤ defines an alias ¤a¤ that is checked to
	      be of type ¤ty¤;
	\item ¤check u: t¤ verifies ¤u¤ has type ¤t¤;
	\item ¤check u¤ provides the type of ¤u¤;
	\item ¤eval u¤ provides the definition of ¤u¤.
\end{itemize}

Below is an overview of the syntax of the terms, including hypothetical
developments. The syntax is partially inspired by that of Coq, OCaml and Lean. Comments are
defined using the keyword \texttt{//}.

\paragraph{Elementary type theory}
\phantom{hello}
\begin{mdln}
// Church construction of natural numbers
def Nat :=
  (N: Type) -> (N -> N) -> N -> N

def z := fun N: Type =>
  fun f: (N -> N), x:N => x
check z: Nat

def succ := fun n: Nat, N: Type =>
  fun f: (N -> N), x: N =>
    f (n N f x)
check succ: Nat -> Nat
\end{mdln}

\paragraph{Universe polymorphism}
\phantom{hello}
\begin{mdln}
def foo.{i,j} : Type (max i j) + 1
  := Type i -> Type j
\end{mdln}

\paragraph{\etde Unification}
\phantom{hello}
\begin{mdln}
def comm := \/x y, x + y = y + x
\end{mdln}

\paragraph{\etun Existential types} as well as other usual types.
\begin{mdln}
def t := E n, n * n - n + 4 = 0
\end{mdln}

\paragraph{\etde General framework for inductive types.} For the moment,
inductive types and their recursors are hardcoded in the kernel.
\begin{mdln}
def inductive Toto :=
  Gur(Nat) | Baba(Toto)
\end{mdln}

\subsection{Toplevel  \members{AuA}}
The \texttt{proost} command, when provided with no argument, is expected to
behave like a toplevel, akin to \texttt{ocaml} or \texttt{coqtop}. There, user
is greeted with a prompt and may enter commands. When provided with existing
file paths, ¤proost¤ intends to typecheck them in order, that is, reading them
as successive inputs in the toplevel. Later features for this include a more
extended notion of ``modules'' \etde where files provide scopes.
\begin{center}
	\fbox{\includegraphics{media/screenshot_toplevel}}
\end{center}


\subsection{LSP \members{VL}}
The \texttt{tilleul} binary provides an implementation of the Language Server
Protocol and will gradually implement more features of it. \etde

\subsection{Parsing \members{AuA}}
The parsing approach is straightforward and
relies on external libraries. The parser is expected to keep adapting to changes
made in the term definitions and unification capability. The parser is
thoroughly tested to guarantee full coverage.


\subsection{Kernel \members{all members}}
The kernel manipulates
\(\lambda\)-terms in the Calculus of Construction and is expected to store and
manage them with a relative level of efficiency. The type theory used to build
the terms will be successively extended with:
\begin{itemize}
	\item abstractions, \(\Pi\)-types, predicative universes with \(\mathsf{Prop}\);
	\item universe polymorphism;
	\item equality types, natural numbers;
	\item \etun \(\Sigma\)-types, etc.;
	\item \etde extraction;
	\item \etde lists, records, accessibility predicate.
\end{itemize}


\subsection{Optimisation \members{ArA LTM}}
Extra care must be put into designing an efficient memory management model for
the kernel, along with satisfactory typing and reduction algorithms.

In particular, the first iteration of the program manipulates directly terms on
the heap, with no particular optimisation: every algorithm is applied soundly
but naively.

A first refactor of the memory model includes using a common memory location for
terms, ensuring invariant like unicity of a term in memory, providing laziness
and storing results of the most expensive functions \emph{(memoizing)}. This
model also provides stronger isolation properties, preventing several memory
pools (\emph{arena} is the technical term used in the project) from interacting
with one another.

\subsection{\etde Unification}\label{sec:unification}
Early versions of the tool may require the user to explicit every type at play.
Successive versions may gradually include unification tools (meta-variables) of
better quality to assist the user and alleviate some of their typing-annotation
burden.

\subsection{Developpement tools \members{VL LTM}}
Tests are mandatory in every part of the project. A tool originally developed by
the Mozilla team was modified to allow for a more precise branch coverage of the
project. The Nix framework is used to automatically build and package the
application as well as generating a docker image and providing developers tools
of the same version.

% TODO elaborate, maybe
\end{document}
