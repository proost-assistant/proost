\documentclass[twocolumn]{article}
\usepackage[british]{babel}
\usepackage{textcase}% provides \MakeTextUppercase (does not impact math mode)
\usepackage{amssymb, amsmath, amsthm, mathrsfs}
\usepackage{fancyhdr}
\usepackage{listings}
\usepackage{bussproofs} % proof tree
\usepackage{xcolor} % may not be loaded here, is already loaded by Beamer
\RequirePackage[a4paper, left=2cm, right=2cm, bottom=3cm, headsep=100pt]{geometry}
\RequirePackage[small]{titlesec} % Taille des sections réduite
\RequirePackage[pdfborderstyle={/S/U/W 0}]{hyperref} % Le paramètre retire les bordures autour des hyperliens

\author{
  Arthur \textsc{Adjedj}\\
  Vincent \textsc{Lafeychine} \and
  Augustin \textsc{Albert} \\
  Lucas \textsc{Tabary-Maujean}
}

\title{\textbf{Proost: specifications}\\
  \large A small proof assistant written in Rust
  \\[1\baselineskip]\normalsize ENS Paris-Saclay
}

%%%%%%%%%%%%%%%%%%%%%%%%%%%%%%%%%%%%%%%%%%%% Listings %%%%%%%%%%%%%%%%%%%%%%%%%%%%%%%%%%%%%%%%%%%%%%
\definecolor{gray}{rgb}{0.5,0.5,0.5}
\definecolor{paleblue}{rgb}{0.7,0.7,1}
\definecolor{darkgray}{rgb}{0.1,0.1,0.1}
\definecolor{dark}{rgb}{0,0,0}
\definecolor{forestgreen}{rgb}{0,0.3,0}
\definecolor{darkred}{rgb}{0.5,0,0}

\lstdefinestyle{default}{
    frame=tb,
    basicstyle=\ttfamily,
    numbers=left,
    numbersep=5pt,
    xrightmargin=1.5em,xleftmargin=1.5em,
    framexleftmargin=1.7em,
    numberstyle=\footnotesize\ttfamily\color{paleblue},
    morestring=[d]{"},
    morestring=[d]{'},
    stringstyle={\color{red!50!brown}},
    keywordstyle={\bfseries\itshape\color{dark}},
    commentstyle=\color{gray},
    texcl, % Back to TeX styling within *inline* comments
    escapechar=`, % escape character for block comments
    %escapebegin=\lst@commentstyle,
    breaklines=true,
    breakatwhitespace=true
}
\lstdefinestyle{proost}{
    style=default,
    numberstyle=\scriptsize\ttfamily\color{paleblue},
    comment=[l]{//},
    alsoletter={.,:,=},
    keywords={fun, return},
    keywordstyle=\color{forestgreen},
    keywords=[4]{define, check, type, :=, .},
    keywordstyle=[4]\color{darkred},
}
\lstnewenvironment{proost}[1][]{\lstset{style=proost, #1}}{}

\lstset{style=proost}
\lstMakeShortInline[columns=flexible]¤
%%%%%%%%%%%%%%%%%%%%%%%%%%%%%%%%%%%%%%%%%%%%%%%%%%%%%%%%%%%%%%%%%%%%%%%%%%%%%%%%%%%%%%%%%%%%%%%%%%%%

%%%%%%%%%%%%%%%%%%%%%%%%%%%%%%%%%%%%%%%%%%%%% headings %%%%%%%%%%%%%%%%%%%%%%%%%%%%%%%%%%%%%%%%%%%%%
\pagestyle{fancy}
\renewcommand{\headrulewidth}{0pt}
\renewcommand{\footrulewidth}{0pt}

\fancyhead{}
\fancyfoot[L]{\small{\reflectbox{\copyright} \the\year{}}}
%%%%%%%%%%%%%%%%%%%%%%%%%%%%%%%%%%%%%%%%%%%%%%%%%%%%%%%%%%%%%%%%%%%%%%%%%%%%%%%%%%%%%%%%%%%%%%%%%%%%

%%%%%%%%%%%%%%%%%%%%%%%%%%%%%%%%%%%%%%%% bussproofs settings %%%%%%%%%%%%%%%%%%%%%%%%%%%%%%%%%%%%%%%
% documentation at https://mirror.ibcp.fr/pub/CTAN/macros/latex/contrib/bussproofs/BussGuide2.pdf
\EnableBpAbbreviations
% Overriding default label style
\def\RL#1{\RightLabel{\footnotesize\bfseries{#1}}}
\def\LL#1{\LeftLabel{\footnotesize\bfseries{#1}}}

% boxed proofs, used to align multiple proofs on a single line
\newenvironment{bprooftree}
  {\leavevmode\hbox\bgroup}
  {\DisplayProof\egroup}

\newcommand{\textr}[1]{{\footnotesize\textbf{\MakeTextUppercase{#1}}}}
% Fixes alignment issues but not without some consistency issues
\def\RL#1{\RightLabel{\makebox[0pt][l]{\textr{#1}}}}
\def\LL#1{\LeftLabel{\makebox[0pt][r]{\textr{#1}}}}
% consistent label (with width)
\def\cRL#1{\RightLabel{\textr{#1}}}
\def\cLL#1{\LeftLabel{\textr{#1}}}
%%%%%%%%%%%%%%%%%%%%%%%%%%%%%%%%%%%%%%%%%%%%%%%%%%%%%%%%%%%%%%%%%%%%%%%%%%%%%%%%%%%%%%%%%%%%%%%%%%%%

%%%%%%%%%%%%%%%%%%%%%%%%%%%%%%%%%%%%%%%%%%%%%% Divers %%%%%%%%%%%%%%%%%%%%%%%%%%%%%%%%%%%%%%%%%%%%%%
\setlength{\columnsep}{20pt} % 10 pt par défaut
%%%%%%%%%%%%%%%%%%%%%%%%%%%%%%%%%%%%%%%%%%%%%%%%%%%%%%%%%%%%%%%%%%%%%%%%%%%%%%%%%%%%%%%%%%%%%%%%%%%%

\begin{document}
\thispagestyle{fancy}
\maketitle

\section{General purpose and functions}
This project aims at providing a small tool for typechecking programs written in
the language of the Calculus of Construction (CoC). This tool shall be
terminal-based and provide both compiler- and toplevel-like capacities and
options

\subsection{Toplevel}
The \texttt{proost}, when provided with no argument, is expected to behave like
a toplevel, akin to \texttt{ocaml}, or \texttt{coqtop}. There, user is greeted with a
prompt and may enter commands of different kinds, for instance:
\begin{itemize}
  \item ¤define a := t¤ defines an alias ¤a¤ that can be used in any following
    command;
  \item ¤check u : t¤ verifies ¤a¤ has type ¤t¤;
  \item ¤type u¤ provides the type of ¤u¤.
\end{itemize}

Where terms are the ones permitted by the kernel expressivity.
More shorthands may be provided later for user convenience.

\subsection{Checker}
When provided with existing file paths, ¤proost¤ intends to typecheck them in
order, that is, reading them as successive inputs in the toplevel. Further
features for this \emph{might} include a more extended notion of ``modules'', where files
may refer to one another and provide scopes. Efforts will likely not be put on
this.

The file extension is \texttt{.mdln}, which is short for \emph{madelaine}.
Madelaine also denotes the language manipulated by users in these files.

\section{Project structure}
Each category of the project is assigned some or all members of the group,
meaning the assignees will \emph{mainly} make progress in the associated categories
and review the corresponding advancements. Any member may regardless contribute to any part
of the development of the tool.

Some specific categories and items will be added a star (\(*\)) or two (\(**\)) to
indicate respectively whether they are late requirements (for the last release)
or extra requirements that will be considered only if there is enough time.


\subsection{Kernel  \hfill\scriptsize all members}
The kernel manipulates \(\lambda\)-terms in the Calculus of Construction and is
expected to store and manage them with a relative level of efficiency.
The type theory used to build the terms will be successively extended with:
\begin{itemize}
  \item abstractions, \(\Pi\)-types, predicative universes with \(\mathsf{Prop}\);
  \item \(*\) \(\Sigma\)-types, equality types, natural numbers;
  \item \(**\) extraction;
  \item \(**\) lists, records, accessibility predicate.
\end{itemize}

\subsubsection{Optimisation \hfill\scriptsize VL LTM}
Extra care must be put into designing an efficient memory management model for
the kernel, along with satisfactory typing and reduction algorithms.


\subsection{Unification \hfill\scriptsize ArA LTM}
Early versions of the tool may require the user to explicit every type at play.
Successive versions may gradually include unification tools (meta-variables)
of better quality to assist the user and alleviate some of their typing-annotation
burden.

% TODO provide some examples


\subsection{Parsing \hfill\scriptsize AuA VL}
The parsing approach is straightforward and relies on external libraries. The
parser is expected to keep adapting to changes made in the term definitions and
unification capability.


\subsection{Language design \hfill\scriptsize all members}
The \texttt{proost} tool is a simple proof assistant and does not provide any
tactics. As new features arrive from extensions of the kernel type theory, the
\emph{madelaine} language must provide convenient shorthands and notations. See
the following examples:

% TODO examples


\subsection{\(*\) LSP \hfill\scriptsize VL}
The tool may provide an implementation of the Language Server Protocol in
order to provide linting and feedback during an editing session.

% TODO elaborate, maybe
\end{document}
