\documentclass[twocolumn]{article}

\usepackage[british]{babel}
\usepackage{fancyhdr}
\RequirePackage[a4paper, left=2cm, right=2cm, bottom=3cm, top=2.3cm, headsep=100pt]{geometry}
\usepackage{header}
\RequirePackage[small]{titlesec} % Taille des sections réduite

\author{
  Arthur \textsc{Adjedj}\\
  Vincent \textsc{Lafeychine} \and
  Augustin \textsc{Albert} \\
  Lucas \textsc{Tabary-Maujean}
}

\title{
  \includegraphics[height=2.5cm]{media/logo}

	\textbf{Proost 0.3.0} \\
	User manual
  \\[1\baselineskip]\normalsize ENS Paris-Saclay
}


%%% headings
\pagestyle{fancy}
\renewcommand{\headrulewidth}{0pt}
\renewcommand{\footrulewidth}{0pt}

\fancyhead{}
\fancyfoot[L]{\small{\reflectbox{\copyright} \the\year{}}}


%%%
\setlength{\columnsep}{20pt} % 10pt is the default

\newcommand{\members}[1]{\texorpdfstring{\hfill\scriptsize #1}{}}

\newcommand{\etun}{{\color{Green} ($\star$)} }
\newcommand{\etde}{{\color{Orange} ($\star\star$)} }


%%%
\begin{document}
\maketitle
\thispagestyle{fancy}

\section{Introduction}
This program provides the user a set of tools to work with \(\lambda\)-terms as
described in the Calculus of Construction (CoC). Through the Curry-Howard
correspondence users can associate a property or theorem they are willing to
prove to a type which can be expressed in CoC. Proving it then corresponds to
constructing a term of that type.

\paragraph{Release 0.1.0} of the project includes a toplevel interface also called
\texttt{proost}, which is the main way users can interact with this piece of
software. For a use directly through the crate APIs, please refer to their
respective documentations.

\paragraph{Release 0.2.0} includes the ability to manipulate universe-polymorphic declarations, a
framework for hardcoding axioms in the code base of the kernel, a better handling
of error locations in the interface and a proof-of-concept implementation of the
LSP protocol.

\paragraph{Release 0.3.0} provides a variety of minor upgrades, including a better printing
of terms, some optimisations in the kernel, many fixes in the parser and the
toplevel, the beginning of a standard library as well as new types, most notably
natural numbers and equality. We would like to thank all new contributors for
their efforts!

\section{Toplevel session}
In the toplevel, users are greeted with a prompt. There, they may enter the
following commands:
\begin{itemize}
	\item ¤import file1 file2¤ typechecks
	      and loads the files in the current environment;
	\item ¤search v¤ looks for the definition of variable ¤v¤;
	\item ¤def a := t¤
	      defines an alias ¤a¤ that can be used in any following command;
	\item ¤def a: ty := t¤ defines an alias ¤a¤ that is checked to
	      be of type ¤ty¤;
	\item ¤check u: t¤ verifies ¤u¤ has type ¤t¤;
	\item ¤check u¤ provides the type of ¤u¤;
	\item ¤eval u¤ provides the normal form of ¤u¤.
\end{itemize}

Optionally, defined terms can be of the form ¤a.{i, j}¤, meaning they are
universe-polymorphic in ¤i¤ and ¤j¤. In that case, they are called
\emph{declarations}. Later, these declarations can be used for creating new
terms, by calling them like ¤a.{n, m}¤, where ¤n¤ and ¤m¤ are well-defined
universe levels.

If the command succeeds, the toplevel returns a green check mark, with an
associated result if there is any. Otherwise, a red cross indicates an error
occurred, next to some details about it. The command is discarded and the user
may enter another command.

The toplevel provides to a certain extent history browsing, either \emph{via}
the up and down arrow keys or some auto-completion from previous commands. The
toplevel also provides partial syntax highlighting, multi-line editing, which
integrates with a detection of the currently-opened parentheses, if any. An
example session is shown in figure \ref{fig:toplevel-example}.

\begin{figure*}
	\centering
	\begin{toplevel}
>>> import std/nat.mdln
VVV
>>> add Zero Zero
XXX ^-^
XXX expected def var := term, [...] eval term, import path_to_file, or search var
>>> eval add Zero Zero
VVV Zero
>>> eval add (add Zero (fun p: Prop -> Prop, x: Prop => p (p x))) Zero
XXX           ^------------------------------------------------^
XXX function (LL Nat => NatRec (LL b: Nat => Nat) 1 (LL Nat => LL Nat => Succ 1)) Zero
XXX expects a term of type Nat, received LL (b: Prop) -> Prop => LL Prop => 2 (2 1):
XXX (a: (b: Prop) -> Prop) -> (b: Prop) -> Prop
	\end{toplevel}
	\label{fig:toplevel-example}
	\caption{Example of an interactive toplevel session}
\end{figure*}

\section{Language}
Language syntax is as such:
\begin{itemize}
	\item Functions (\(\lambda\)-abstractions) are defined with the keyword ¤fun¤:
	      ¤fun x: A => u¤, ¤fun x y: A => u¤ (both ¤x¤ and ¤y¤ are of type ¤A¤),
	      ¤fun x: A, y: B => u¤ (multiple arguments, where ¤B¤ may depend on ¤x¤);

	\item Dependent function types (\(\Pi\)-types) are defined with a pair of parentheses
	      before an arrow, as in: ¤(x: A) -> B¤, ¤(x y: A) -> B¤ or
	      ¤(x: A, y: B) -> C¤ (where the distinctions are similar to the previous
	      item. Additionally, there is some usual syntactic sugar when the output
	      type does not mention the input argument, which corresponds to usual
	      function types: ¤A -> B¤, ¤A -> B -> C¤ (right-associativity);

	\item Function application of two terms ¤u¤ and ¤v¤ is simply written ¤u v¤,
	      and is left-associative when there are multiple arguments;

	\item Variables are regular strings, which may only correspond to bound
	      variables or previously defined terms (or declarations as explained in the
	      previous section);

	\item The type of propositions and higher-order types are written ¤Prop¤ and
	      ¤Type i¤, as usual. One may also refer to the universes in hierarchy through
	      the ¤Sort¤ keyword as follows: ¤Sort 0 = Prop¤ and ¤Sort n + 1 = Type n¤.
\end{itemize}

The syntax of universe levels is the following:
\begin{itemize}
	\item numeric constants and level variables defined in the definition of a
	      declaration are valid universe level;
	\item if ¤u¤ is a valid level, ¤u + n¤ is valid, where ¤n¤ is a numeric
	      constant;
	\item if ¤u¤ and ¤v¤ are valid levels, ¤max u v¤ and ¤imax u v¤ (impredicative
	      maximum) are valid levels.
\end{itemize}

\section{Axioms}
There is no simple way to browse the set of axioms or their recursors for the
moment. This should be made easier in future releases. The general current
syntax is, for an inductive type ¤Foo¤, to have a declaration ¤Foo_rec¤ which
corresponds to its recursor, and other appropriately-named functions as its
constructors.

Please refer to the code presented in ¤example/¤ and ¤std/¤ for concrete
examples.

\end{document}
