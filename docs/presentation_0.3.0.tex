\documentclass[12pt, aspectratio=169]{beamer}

\usepackage[utf8]{inputenc}
\usepackage[T1]{fontenc}
\usepackage[french]{babel}

\usepackage{graphicx}
\usepackage{xcolor}
\usepackage{listings}
\usepackage{lstautogobble}


\usepackage{amsmath}
\usepackage{amssymb}
\usepackage{ebproof}
\usepackage{tikz}

\definecolor{darkgreen}{HTML}{18A100}
\definecolor{background}{rgb}{0.95,0.95,0.92}
\definecolor{codegreen}{rgb}{0,0.6,0}
\definecolor{codegray}{rgb}{0.5,0.5,0.5}
\definecolor{codepurple}{rgb}{0.58,0,0.82}

\lstset{
    literate =
    {á}{{\'a}}1
    {à}{{\`a}}1
    {Â}{{\^A}}1
    {ã}{{\~a}}1
    {é}{{\'e}}1
    {É}{{\'E}}1
    {è}{{\`e}}1
    {ê}{{\^e}}1
    {ó}{{\'o}}1
    {ü}{{\^u}}1
    {î}{{\^i}}1
    {ç}{{\c{c}}}1
    {œ}{{\oe}}1
    {Œ}{{\OE}}1
}

\lstset{
    autogobble=true,
    showstringspaces=false,
    basicstyle=\small\ttfamily,
    backgroundcolor=\color{background},
    keywordstyle=\color{codepurple},
    commentstyle=\color{codegreen},
    showtabs=False,
    stringstyle=\color{orange},
    frame=single,
    escapeinside={(*}{*)},
    tabsize=4,
    breaklines=true
}

\lstdefinelanguage{mdln}{
    keywords={def, check, eval, Sort, Prop, Type, import, fun}
}


\title[Rendu final : Proost]{Rendu final de projet génie logiciel : Proost}
\author[Projet génie logiciel]{
    \normalsize
    Arthur~Adjedj \and 
    Augustin~Albert \and
    Vincent~Lafeychine \and
    Lucas~Tabary-Maujean \and
    Jean~Abou-Samra \and
    Tanguy~Bozec \and
    Antonin~Bretagne \and
    Vivien~Ducros \and
    Antoine~Guilmin-Crépon \and
    Balthazar~Patiachvili \and
    Thomas~Varin
}
\date{12 janvier 2023}
\institute[]{ENS Paris-Saclay}
\logo{
    \begin{picture}(0,0)
        \put(-130,-50) {
            \begin{tikzpicture}
                \node[opacity=0.5] (0) {\includegraphics[scale=1.5]{media/logo.png}};
            \end{tikzpicture}
        }
    \end{picture}
}

\usetheme{Warsaw}
\useoutertheme{shadow}
\useoutertheme{infolines}

\AtBeginSection[]
{
    \begin{frame}[noframenumbering]
        \frametitle{Sommaire}
        \tableofcontents[sectionstyle=show/shaded, subsectionstyle=show/shaded]
    \end{frame}
}

\newcommand{\jump}[1][1]{\vspace{1\baselineskip}}

\begin{document}

    \maketitle

    \section{Présentation générale}

        \begin{frame}[fragile]{Présentation du projet}
        
            \begin{block}{Qu'est-ce que Proost ?}
                
                Proost est un assistant de preuve écrit en Rust basé sur le calcul des constructions. À terme, il est prévu que cela soit remplacé par le calcul observationnel des constructions $\left(\mathrm{CC}^{\mathrm{obs}+}\right)$.

            \end{block} \pause

            \begin{block}{Quel est le langage de preuve ?}
                
                Il s'agit de Madelaine, un nouveau langage inspiré de Coq, Lean, \dots

            \end{block} \pause

            \begin{exampleblock}{Exemple de code}
                
                \begin{lstlisting}[language=mdln]
                    def And: Prop -> Prop -> Prop := fun A B: Prop => (C: Prop) -> (A -> B -> C) -> C
                \end{lstlisting}

            \end{exampleblock}

        \end{frame}

        \begin{frame}{Philosophie du projet}
            
            \begin{alertblock}{Points importants}
                
                \begin{itemize}
                    \item Isolation du kernel pour assurer la correction de l'implémentation \pause
                    \item API d'interface avec le noyau \pause
                    \item Expérience utilisateur agréable (LSP, coloration syntaxique, \dots) \pause
                    \item Optimisation mémoire et temporelle du type-checking
                \end{itemize}

            \end{alertblock}

        \end{frame}

    \section{Description de la théorie implémentée}

        \begin{frame}[fragile]{Univers}
            
            \begin{block}{Hierarchie prédicative}
        
                Type 0 : Type 1 : \dots \pause

                \begin{center}
                    \begin{prooftree}
                        \hypo{\Gamma \vdash A : \mathtt{Type}\ u}
                        \hypo{\Gamma, x : A \vdash B : \mathtt{Type}\ v}
                        \infer2{\Gamma \vdash  \Pi(x : A).\;  B: \mathtt{Type}\ \max u\ v}
                    \end{prooftree}
                \end{center}
                
            \end{block}

            \begin{block}{Univers imprédicatif proof-irrelevant : Prop}
                
                Prop : Type 0 \pause

                \begin{center}
                    \begin{prooftree}
                        \hypo{\Gamma \vdash A : s}
                        \hypo{\Gamma, x : A \vdash B : \mathtt{Prop}}
                        \infer2{\Gamma \vdash  \Pi(x : A).\  B: \mathtt{Prop}}
                    \end{prooftree}
                \end{center}

            \end{block}

        \end{frame}

        \begin{frame}[fragile]{Polymorphisme d'univers}
            
            \begin{block}{Hierarchie contenant Prop}
               
                \begin{tabular}{llll}
                    \texttt{Prop} :& \texttt{Type 0} :& \texttt{Type 1} :& \dots \\
                    \texttt{Sort 0} :& \texttt{Sort 1} :& \texttt{Sort 2} :& \dots
                \end{tabular}
                
            \end{block}

            \begin{exampleblock}{Example}
                
                \begin{lstlisting}[language=mdln]
                    def id.{u} := fun A : Sort u, x : A => x
                \end{lstlisting}

            \end{exampleblock}

        \end{frame}

        \begin{frame}[fragile]{Types inductifs variés}
            
            \begin{exampleblock}{Exemples de types inductifs implémentés}
                
                \begin{itemize}
                    \item \lstinline{False} 
                    \item \lstinline{True}
                    \item \lstinline{Nat}
                    \item \lstinline{Eq}
                    \item \dots
                \end{itemize}

            \end{exampleblock} \pause

            \begin{exampleblock}{Exemples}
                
                \begin{lstlisting}[language=mdln]
                    def add := fun x : Nat => Nat_rec.{1} (fun n: Nat => Nat) x (fun n n:Nat => Succ n)
                    def two := Succ (Succ Zero)
                    def four := Succ (Succ two)
                    check Refl.{1} Nat four : Eq.{1} Nat four (add two two)
                \end{lstlisting}

            \end{exampleblock}

        \end{frame}

        \begin{frame}{Égalité}
            
            \begin{block}{Actuellement}
                
                Égalité intensionnelle à la MLTT

            \end{block} \pause

            \begin{alertblock}{Dans le futur}
                
                Égalité observationnelle à la $\mathrm{CC}^{\mathrm{obs}+}$

            \end{alertblock}

        \end{frame}

    \section{Caractéristiques de Proost}

        \begin{frame}{Toplevel}
            
            \begin{block}{Caractéristiques}
                
                \begin{itemize}
                    \item Accessible via l'invite de commandes
                    \item Coloration syntaxique
                    \item Localisation précise et visuelle des erreurs
                \end{itemize}

            \end{block} \pause

            \begin{block}{Toplevel}
                
                \begin{center}
                    \includegraphics[scale=1.3]{media/screenshot_toplevel.png}
                \end{center}

            \end{block}

        \end{frame}

        \begin{frame}{Type-checking}
            
            \begin{block}{Caractéristiques}
                
                \begin{itemize}
                    \item Lourde mémoïsation des termes et des computations sur ces derniers (WHNF, substitution, var-shifting, \dots) \pause
                    \item Système d'arène pour gérer efficacement les lifetimes des termes
                \end{itemize}

            \end{block}

        \end{frame}

        \begin{frame}{LSP}
            
            \begin{block}{Description}
                
                Le LSP s'appelle <<~Tilleul~>> et se compose :

                \begin{itemize}
                    \item d'une partie I/O répartissant les requêtes~;
                    \item d'une partie logique traitant les requêtes.
                \end{itemize}

            \end{block} \pause

            \begin{block}{Avancement}
                
                Une preuve de concept est actuellement disponible sur la partie I/O. Cependant la partie logique n'a pas encore été commencée.

            \end{block}

        \end{frame}

    \section{Pour la suite}

        \begin{frame}{Projets pour la suite}
            
            \begin{block}{Dans un futur proche}
                
                \begin{itemize}
                    \item Égalité observationnelle
                    \item LSP entièrement fonctionnel
                    \item Espaces de noms importés
                \end{itemize}

            \end{block} \pause

            \begin{block}{Dans un second temps}
                
                \begin{itemize}
                    \item Types inductifs
                    \item Unification
                \end{itemize}

            \end{block}

        \end{frame}

\end{document}