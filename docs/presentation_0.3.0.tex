\documentclass[12pt, aspectratio=169]{beamer}

\usepackage[utf8]{inputenc}
\usepackage[T1]{fontenc}
\usepackage[french]{babel}
\usepackage{hyperref}

\usepackage{graphicx}
\usepackage{xcolor}
\usepackage{listings}
\usepackage{lstautogobble}

\usepackage{amsmath}
\usepackage{amssymb}
\usepackage{ebproof}

% For top-level symbols
\usepackage{arevmath}

\definecolor{darkgreen}{HTML}{18A100}
\definecolor{background}{rgb}{0.95,0.95,0.92}
\definecolor{codegreen}{rgb}{0,0.5,0}
\definecolor{codegray}{rgb}{0.5,0.5,0.5}
\definecolor{codered}{rgb}{0.5,0,0.0}
\definecolor{background1}{RGB}{248, 243, 237}
\definecolor{foreground1}{RGB}{118, 66, 54}

\setbeamercolor{palette primary}{bg=background1, fg=foreground1}
\setbeamercolor{palette secondary}{bg=background1,fg=foreground1}
\setbeamercolor{palette tertiary}{bg=background1,fg=foreground1}
\setbeamercolor{palette quaternary}{bg=background1,fg=foreground1}
\setbeamercolor{structure}{fg=foreground1}
\setbeamercolor{section in toc}{fg=foreground1}
\setbeamercolor{subsection in head/foot}{bg=background1,fg=foreground1}
\setbeamercolor{alerted text}{fg=codered}
\setbeamercolor{example text}{fg=codegreen}

\usetheme{Goettingen}
\useinnertheme{circles}


\lstset{
    literate =
    {á}{{\'a}}1
    {à}{{\`a}}1
    {Â}{{\^A}}1
    {ã}{{\~a}}1
    {é}{{\'e}}1
    {É}{{\'E}}1
    {è}{{\`e}}1
    {ê}{{\^e}}1
    {ó}{{\'o}}1
    {ü}{{\^u}}1
    {î}{{\^i}}1
    {ç}{{\c{c}}}1
    {œ}{{\oe}}1
    {Œ}{{\OE}}1
}

\lstset{
    autogobble=true,
    showstringspaces=false,
    basicstyle=\small\ttfamily,
    backgroundcolor=\color{background1},
    keywordstyle=\color{codered},
    commentstyle=\color{codegreen},
    showtabs=False,
    stringstyle=\color{orange},
    frame=single,
    escapeinside={(*}{*)},
    mathescape=true,
    tabsize=4,
    breaklines=true
}

\lstdefinelanguage{mdln}{
    keywords={def, check, eval, Sort, Prop, Type, import, fun}
}


\title[Rendu final : Proost]{Rendu final de projet génie logiciel : Proost}
\author[Projet génie logiciel]{
    \normalsize
    Arthur~Adjedj \and
    Augustin~Albert \and \\
    Vincent~Lafeychine \and
    Lucas~Tabary-Maujean \and \\ \vspace{0.2cm}
    {\footnotesize
      Jean~Abou-Samra \and
      Tanguy~Bozec \and
      Antonin~Bretagne \and
      Vivien~Ducros \and
      Antoine~Guilmin-Crépon \and
      Balthazar~Patiachvili}
}
\date{12 janvier 2023}
\institute[]{ENS Paris-Saclay}

\addtobeamertemplate{footline}{%
  \setlength\unitlength{1ex}%
  \begin{picture}(0,0)
    % \put{} defines the position of the frame
    \put(145,-5){\makebox(0,0)[bl]{
    \includegraphics[scale=0.7]{media/logo.png}
    }}%
  \end{picture}%
}{}

\AtBeginSection[]
{
    \begin{frame}[noframenumbering]
        \frametitle{Sommaire}
        \tableofcontents[sectionstyle=show/shaded, subsectionstyle=show/shaded]
    \end{frame}
}

\newcommand{\jump}[1][1]{\vspace{1\baselineskip}}

\begin{document}
    \beamertemplatenavigationsymbolsempty
    \maketitle

    \section{Présentation générale}

        \begin{frame}[fragile]{Présentation du projet}

            \begin{block}{Qu'est-ce que Proost ?}

                Proost est un assistant de preuve écrit en Rust basé sur le calcul des constructions. À terme, il est prévu que cela soit remplacé par le calcul observationnel des constructions $\left(\mathsf{CC}^{\mathrm{obs}+}\right)$.

            \end{block} \pause

            \begin{block}{Quel est le langage de preuve ?}

                Il s'agit de Madeleine, un nouveau langage inspiré de Coq, Lean, \dots

            \end{block} \pause

            \begin{exampleblock}{Exemple de code}

                \begin{lstlisting}[language=mdln]
                    def And: Prop -> Prop -> Prop :=
                        fun A B: Prop => (C: Prop) -> (A -> B -> C) -> C
                \end{lstlisting}

            \end{exampleblock}

        \end{frame}

        \begin{frame}{Philosophie du projet}

            \begin{alertblock}{Points importants}

                \begin{itemize}
                    \item Isolation du kernel pour assurer la correction de l'implémentation \pause
                    \item API d'interface avec le noyau \pause
                    \item Expérience utilisateur agréable (LSP, coloration syntaxique, toplevel) \pause
                    \item Optimisation en temps et en mémoire du type-checking
                \end{itemize}

            \end{alertblock}

        \end{frame}

    \section{Description de la théorie implémentée}

        \begin{frame}[fragile]{Univers}

            \begin{block}{Hiérarchie prédicative}

                Type 0 : Type 1 : \dots

                \begin{center}
                    \begin{prooftree}
                        \hypo{\Gamma \vdash A : \mathtt{Type}\ u}
                        \hypo{\Gamma, x : A \vdash B : \mathtt{Type}\ v}
                        \infer2{\Gamma \vdash  \Pi(x : A).\;  B: \mathtt{Type}\ \max u\ v}
                    \end{prooftree}
                \end{center}

            \end{block}

            \pause

            \begin{block}{Univers imprédicatif proof-irrelevant : Prop}

                Prop : Type 0

                \begin{center}
                    \begin{prooftree}
                        \hypo{\Gamma \vdash A : s}
                        \hypo{\Gamma, x : A \vdash B : \mathtt{Prop}}
                        \infer2{\Gamma \vdash  \Pi(x : A).\  B: \mathtt{Prop}}
                    \end{prooftree}
                \end{center}

            \end{block}

        \end{frame}

        \begin{frame}[fragile]{Polymorphisme d'univers}

            \begin{block}{Hiérarchie contenant Prop}

                \begin{tabular}{lclclcl}
                    \texttt{Prop} &:& \texttt{Type 0} &:& \texttt{Type 1} &:& \dots \\
                    \texttt{Sort 0} &:& \texttt{Sort 1} &:& \texttt{Sort 2} &:& \dots
                \end{tabular}

            \end{block}

            \vfill

            \begin{exampleblock}{Exemple}

                \begin{lstlisting}[language=mdln]
                    def id.{u} := fun A: Sort u, x: A => x
                \end{lstlisting}

            \end{exampleblock}

        \end{frame}

        \begin{frame}[fragile]{Types inductifs variés}

            \begin{exampleblock}{Types inductifs implémentés}

                \begin{itemize}
                    \item \lstinline{False} et \lstinline{True}
                    \item \lstinline{Nat}, \lstinline{Eq}, \dots
                \end{itemize}

            \end{exampleblock} \pause

            \begin{exampleblock}{Exemple d'utilisation de \lstinline{Eq} et de \lstinline{Nat}:}

                \begin{lstlisting}[language=mdln]
                    def add := fun x: Nat => Nat_rec.{1}
                        (fun _: Nat => Nat) x (fun _ n: Nat => Succ n)

                    def two := Succ (Succ Zero)
                    def four := Succ (Succ two)

                    check Refl.{1} Nat four:
                        Eq.{1} Nat four (add two two)
                \end{lstlisting}

            \end{exampleblock}

        \end{frame}

        \begin{frame}{Égalité}

            \begin{block}{Actuellement}

                Égalité intensionnelle à la MLTT

            \end{block}

            \vfill
            \pause

            \begin{alertblock}{Dans le futur}

                Égalité observationnelle à la $\mathrm{CC}^{\mathrm{obs}+}$

            \end{alertblock}

        \end{frame}

    \section{Caractéristiques de Proost}

        \begin{frame}{Noyau}

            \begin{block}{Caractéristiques}

                \begin{itemize}
                    \item Mémoïsation quasi-systématique des
                        calculs sur les termes (WHNF, substitution,
                        \textit{var-shifting}, \dots): soit dans des \textit{hashmaps},
                        soit dans les termes eux-mêmes;\pause
                    \item rendu plausible par la mise en place d'une
                        \emph{arène} (cf. implémentation de DDB) telle que: \pause
                    \begin{itemize}
                        \item chaque terme est unique dans l'arène; \pause
                        \item chaque arène est associée de manière unique à une
                            \textit{lifetime} (garanties statiques, \dots) \pause
                    \end{itemize}

                    \item API \textit{safe} utilisable seule et avec des
                        syntaxes simples (en interne, pour la \textbf{construction}
                        de termes: type clôture).
                \end{itemize}

            \end{block}

        \end{frame}

        \begin{frame}[fragile]{Toplevel}

            \begin{block}{Caractéristiques}

                \begin{itemize}
                    \item Accessible via l'invite de commandes \pause
                    \item Coloration syntaxique \pause
                    \item Localisation précise et visuelle des erreurs
                \end{itemize}

            \end{block} \pause

            \begin{block}{Exemple}

                \begin{lstlisting}[language=mdln]
                    > check add Zero (fun id x: Prop => x)
                    (*$\ballotx$*)       ^----------------------------^
                    (*$\ballotx$*) function ((*$\lambda$*)Nat => (((NatRec.{1}) ((*$\lambda$*) Nat => Nat)) 1)
                    (*$\ballotx$*)         ((*$\lambda$*)Nat => (*$\lambda$*)Nat => Succ 1)) Zero
                    (*$\ballotx$*)     expects a term of type Nat,
                    (*$\ballotx$*)     received (*$\lambda$*) Prop => 1: Prop -> Prop
                \end{lstlisting}

            \end{block}

        \end{frame}

        \begin{frame}{LSP}

            \begin{block}{Description}

                Le LSP s'appelle Tilleul et se compose : \pause

                \begin{itemize}
                    \item d'une partie I/O répartissant les requêtes~; \pause
                    \item d'une partie logique traitant les requêtes.
                \end{itemize}

            \end{block}

            \vspace{.3cm}
            \pause

            \begin{block}{Avancement}

                Une preuve de concept est actuellement disponible sur la partie I/O. Cependant la partie logique n'a pas encore été commencée.

            \end{block}

        \end{frame}

        \begin{frame}{Fonctionnement du projet}

            \begin{block}{Utilisation de GitLab}

                Le projet est disponible sur le \href{https://gitlab.crans.org/loutr/proost}{GitLab du CRANS}. \pause

                \vspace{.3cm}

                Nous effectuons du \href{https://gitlab.crans.org/loutr/proost/-/merge_requests/60}{\textit{peer-reviewing}} à chaque \textit{merge request}.

            \end{block}

            \vfill

            \begin{block}{Infrastructure}

                À chaque modification du projet, une \textit{pipeline} s'exécute et: \pause
                \begin{itemize}
                    \item vérifie le formatage et effectue des analyses statiques~; \pause
                    \item génère des \href{https://perso.crans.org/v-lafeychine/proost/coverage/main/index.html}{couvertures de tests}~; \pause
                    \item génère une \href{https://perso.crans.org/v-lafeychine/proost/doc/proost/}{documentation}.
                \end{itemize}

            \end{block}

        \end{frame}

    \section{Pour la suite}

        \begin{frame}{Projets pour la suite}

            \begin{block}{Dans un futur proche}

                \begin{itemize}
                    \item Égalité observationnelle \pause
                    \item LSP entièrement fonctionnel \pause
                    \item Espaces de noms importés
                \end{itemize}

            \end{block}

            \vfill
            \pause

            \begin{block}{Dans un second temps}

                \begin{itemize}
                    \item Types inductifs \pause
                    \item Unification
                    \item Vérification du kernel avec \href{https://github.com/xldenis/creusot}{Creusot}.
                \end{itemize}

            \end{block}

        \end{frame}

\end{document}
